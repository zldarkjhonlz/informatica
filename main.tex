%% %%%%%%%%%%%%%%%%%%%%%%%%%%%%%%%%%%%%%%%%%%%%%%%%%
%% Template for a conference paper, prepared for the
%% Food and Resource Economics Department - IFAS
%% UNIVERSITY OF FLORIDA
%% %%%%%%%%%%%%%%%%%%%%%%%%%%%%%%%%%%%%%%%%%%%%%%%%%
%% Version 1.0 // November 2019
%% %%%%%%%%%%%%%%%%%%%%%%%%%%%%%%%%%%%%%%%%%%%%%%%%%
%% Ariel Soto-Caro
%%  - asotocaro@ufl.edu
%%  - arielsotocaro@gmail.com
%% %%%%%%%%%%%%%%%%%%%%%%%%%%%%%%%%%%%%%%%%%%%%%%%%%
\documentclass[11pt]{article}
\usepackage{UF_FRED_paper_style}

\usepackage{lipsum}  %% Package to create dummy text (comment or erase before start)

%% ===============================================
%% Setting the line spacing (3 options: only pick one)
% \doublespacing
% \singlespacing
\onehalfspacing
%% ===============================================

\setlength{\droptitle}{-5em} %% Don't touch

% %%%%%%%%%%%%%%%%%%%%%%%%%%%%%%%%%%%%%%%%%%%%%%%%%%%%%%%%%%
% SET THE TITLE
% %%%%%%%%%%%%%%%%%%%%%%%%%%%%%%%%%%%%%%%%%%%%%%%%%%%%%%%%%%

% TITLE:
\title{nacimiento de la computación}

% AUTHORS:
\author{Jhon Ricardo Rios\\% Name author
    \href{mailto:jhon.rios1@udea.edu.co}{\texttt{jhon.rios1@udea.edu.co}} %% Email author 1 
    }
% DATE:
\date{\today}

% %%%%%%%%%%%%%%%%%%%%%%%%%%%%%%%%%%%%%%%%%%%%%%%%%%%%%%%%%%
% %%%%%%%%%%%%%%%%%%%%%%%%%%%%%%%%%%%%%%%%%%%%%%%%%%%%%%%%%%
\begin{document}
% %%%%%%%%%%%%%%%%%%%%%%%%%%%%%%%%%%%%%%%%%%%%%%%%%%%%%%%%%%
% %%%%%%%%%%%%%%%%%%%%%%%%%%%%%%%%%%%%%%%%%%%%%%%%%%%%%%%%%%
% ABSTRACT
% %%%%%%%%%%%%%%%%%%%%%%%%%%%%%%%%%%%%%%%%%%%%%%%%%%%%%%%%%%
% %%%%%%%%%%%%%%%%%%%%%%%%%%%%%%%%%%%%%%%%%%%%%%%%%%%%%%%%%%
{\setstretch{.8}
\maketitle
% %%%%%%%%%%%%%%%%%%

}

% %%%%%%%%%%%%%%%%%%%%%%%%%%%%%%%%%%%%%%%%%%%%%%%%%%%%%%%%%%
% %%%%%%%%%%%%%%%%%%%%%%%%%%%%%%%%%%%%%%%%%%%%%%%%%%%%%%%%%%
% BODY OF THE DOCUMENT
% %%%%%%%%%%%%%%%%%%%%%%%%%%%%%%%%%%%%%%%%%%%%%%%%%%%%%%%%%%
% %%%%%%%%%%%%%%%%%%%%%%%%%%%%%%%%%%%%%%%%%%%%%%%%%%%%%%%%%%

% --------------------
\section{Introducci{\`o}n}
% --------------------

\begin{center}
Desde que se crearon las matemáticas han sido útiles para el vivir diario y siempre han habido
personajes que han hecho grandes aportes a este campo tales como Pitágoras, Isaac Newton, Alan Turing, por mencionar algunos,
con las matemáticas se han logrado resolver muchos problemas pero habían algunos que no eran posibles de resolver ya que las matemáticas no eran consistentes y se evidencio que los fundamentos en los cuales está se basa no estaban bien definidos, a esto se le conoce como la crisis de los fundamentos, en este texto veremos en que consiste la crisis, que desencadeno este hecho para la computación y quienes fueron los encargados de solucionar dicho problema.
\end{center}

% --------------------
\section{Desarrollo}

\begin{center}

El descubrimiento de diferentes infinitos en matemáticas pone en duda su existencia ya que las concesiones que se tenìan en ese tiempo conllevaban a contradicciones.
(paradojas) entre ellas se encuentra la de Cantor: 
"Teorema A: dado un número cardinal, siempre es posible determinar otro mayor.
Teorema B: Existe un número cardinal mayor que todos los demás"

estos hechos desencadenaron que surgieran nuevos movimientos filosoficos como:


\begin{enumerate}
    \item el logicismo por Russel
    \item el intuicionismo por Brouwer, Weyl, Borel, Kronecker y Poincaré
    \item el formalismo por Hilbert
\end{enumerate}

logismo\\
pretendia reconstruir la teoría de conjuntos con lo cual lograría que no aparecieran nuevas paradojas pero no garantizaba que no aparecieran a futuro pero este movimiento no pudo dar soluciòn al problema por lo cual surge el formalismo.\\

Formalismo\\
Hilbert pretendia elegir axiomas los cuales no fueran a producir contradicciones, con lo que buscaba que las matemáticas fueran consistentes pero debido a los aportes de Godel se demostro que no es posible una prueba de la consistencia absoluta.\\

Intuicionismo\\
su propuesta era reconstruir la matemática sin usar el infinito pero gracias a que Hilbert demostro que el infinito actual no origina contradicciones la idea se desmorono.\\

\cite{fernandez1988crisis}
\cite{hitt2013infinito}

\end{center}

% --------------------
\section{Conclusiones}
% --------------------

\begin{center}
A partir de esto se origina la idea de la computación ya que se puede emplear con un conjunto de pasos con un orden y que sean consisos, es decir que no generen contradicciones.
\end{center}


% %%%%%%%%%%%%%%%%%%%%%%%%%%%%%%%%%%%%%%%%%%%%%%%%%%%%%%%%%%
% %%%%%%%%%%%%%%%%%%%%%%%%%%%%%%%%%%%%%%%%%%%%%%%%%%%%%%%%%%
% REFERENCES SECTION
% %%%%%%%%%%%%%%%%%%%%%%%%%%%%%%%%%%%%%%%%%%%%%%%%%%%%%%%%%%
% %%%%%%%%%%%%%%%%%%%%%%%%%%%%%%%%%%%%%%%%%%%%%%%%%%%%%%%%%%
\medskip

\bibliography{references.bib} 

\newpage

\end{document}